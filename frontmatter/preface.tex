\begin{preface}
    学位论文是研究生科研工作成果的集中体现,是研究生培养工作的重要环节,是申请博士、硕士学位的主要依据,也是社会重要的文献资料。

    论文的选题和研究内容,应对学术发展、经济建设和社会进步有一定的理论意义和现实意义。
    论文应具有系统性和完整性,且应当在导师指导下由学位申请人独立完成。
    论文写作和学位申请过程中应当恪守学术道德和学术规范,严格遵守国家相关的法律、法规及本校相关学术规范的相关规定,尊重知识产权,严谨治学,维护科学诚信。

    博士学位论文表明作者在本学科或者专业领域掌握坚实而全面的基础理论和系统深入的专门知识,学术学位申请人应当具有独立从事学术研究工作的能力,专业学位申请人应当具有独立承受专业实践工作的能力;学术学位申请人应当在学术研究领域做出创新性成果,专业学位申请人应当在专业实践领域做出创新性成果。

    硕士学位论文表明作者在本学科或者专业领域掌握坚实的基础理论和系统的专门知识,学术学位申请人应当具有从事学术研究工作的能力,专业学位申请人应当具有承担专业实践工作的能力。

    为了提高研究生学位论文质量,进一步促进我校研究生学位论文的规范化,我们在原《南开大学研究生学位论文写作规范(试行)》的基础上,参考GB/T 7713.1-2006《学位论文编写规则》等相关文件,对《写作规范》做了进一步修订,供申请学位的研究生参考,以利于学位论文的撰写、收藏、存储、加工、检索和利用。

    本规范适用于印刷型、缩微型、电子版、网络版等形式的学位论文。
    同一论文的不同载体形式,其内容和格式应完全一致。

    本规范是一个指导性规范,各一级学科学位评定分委员会可在参考本规范基础上,针对不同学科的学位类型的培养要求,分别制订相应的学位论文写作的具体规范(含外语类学位论文)。

    本规范共分四章,分别为1 内容要求,2 格式要求,3 书写要求,4 排版及印刷要求,并有附录A-G。
\end{preface}
