% !TeX encoding = UTF-8
% !TeX program = xelatex

\documentclass[headingmode=2]{nkthesis}

\usepackage{float}  % 用于控制浮动体位置,仅用于本示例文档,请根据实际需要添加或删除

\nktset{
    中图分类号 = {},
    学校代码 = 10055,
    UDC = {},
    题名页/密级 = 公开,
    题名页/论文类别 = 博士,  % 可选项:硕士/博士/硕士专业/博士专业
    作者信息/论文类别 = 博士,  % 可选项:博士/学历硕士/专业学位硕士/同等学力硕士
    题名页/论文题目字号 = 3,  % 字号最大为2,可根据标题长度适当缩小标题字号
    论文题目(中文) = 现代绿色化学中的物理有机问题,
    论文副标题 = {——酵母菌催化反应机理和咪唑类离子液体酸度的研究},
    论文题目(英文) = Physical Organic Concerns in Green Chemistry: Mechanistic Aspects of Baker's Yeast Mediated Reduction and Measurements of Acidity of Imidazolium Ionic Liquids,
    论文作者 = 某某,
    指导教师 = 某某某,
    指导教师职称 = 教授,
    申请学位 = 理学博士,
    培养单位 = 化学学院,
    一级学科 = 化学,
    二级学科 = 有机化学,
    研究方向 = 物理有机化学,
    答辩委员会主席 = ×××,
    评阅人 = ××× ××× ×××,
    论文完成时间 = 二〇〇七年五月,
    论文编号 = {},
    非公开论文/申请密级 = 公开,  % 申请密级(公开/限制/秘密/机密),如果不为公开,需要填写下面注释掉的字段
    % 非公开论文/保密期限/起始日期 = {},
    % 非公开论文/保密期限/结束日期 = {},
    % 非公开论文/审批表编号 = {},
    % 非公开论文/批准日期 = {},
    授权书/签字日期 = 2025年4月2日,
    学号 = 1234567890,
    答辩日期 = 2024年5月20日,
    联系电话 = 13300000000,
    电子邮箱 = user@example.com,
    通讯地址 = 天津市南开区卫津路94号,300071,
    作者信息/备注 = {},
}
\addbibresource{main.bib}

\begin{document}

% 以下页面可根据需要(匿名/非匿名等)选择使用
\titlepage
\anonymoustitlepage
\declarationpage
\authorizationpage

\frontmatter

\begin{abstract}
    中文摘要是论文内容的简要陈述,是一篇具有独立性和完整性的短文,一般以第三人称语气写成,不加评论和补充的解释。
    摘要具有自含性,即不阅读论文的全文,就能获得必要的信息。
    摘要的内容应包括与论文等同的主要信息,供读者确定有无必要阅读全文,也可供二次文献采用。
    摘要一般应说明研究工作的目的、研究方法、研究成果和结论,要突出本论文的创造性成果。

    中文摘要力求语言精炼准确,一般字数为500-800字,篇幅以一页为宜。如需要,字数可以略多。

    用外文撰写学位论文时,须有详细中文摘要。

    摘要中不可出现图、表、化学方程式、非公知公用的符号和术语。

    关键词在摘要内容后另起一行标明,一般3-5个,之间用分号分开。
    关键词是为了便于做文献索引和检索工作而从论文中选取出来用以表示全文主题内容信息的单词或术语,应体现论文特色,具有语义性,在论文中有明确出处。
    应尽量采用《汉语主题词表》或各专业主题词表提供的规范词。
\end{abstract}

\begin{keywords}
    关键词1;关键词2;关键词3;关键词4;关键词5
\end{keywords}

\begin{enabstract}
    The abstract should correspond to the content of the Chinese summary.
    It should typically contain no fewer than 300 English words and ideally fit within one page.
    If necessary, the word count can be slightly higher.
\end{enabstract}

\begin{enkeywords}
    Keyword1; Keyword2; Keyword3; Keyword4; Keyword5
\end{enkeywords}

\begin{preface}
    学位论文是研究生科研工作成果的集中体现,是研究生培养工作的重要环节,是申请博士、硕士学位的主要依据,也是社会重要的文献资料~\cite{REF00000001}。

    论文的选题和研究内容,应对学术发展、经济建设和社会进步有一定的理论意义和现实意义。
    论文应具有系统性和完整性,且应当在导师指导下由学位申请人独立完成。
    论文写作和学位申请过程中应当恪守学术道德和学术规范,严格遵守国家相关的法律、法规及本校相关学术规范的相关规定,尊重知识产权,严谨治学,维护科学诚信。

    博士学位论文表明作者在本学科或者专业领域掌握坚实而全面的基础理论和系统深入的专门知识,学术学位申请人应当具有独立从事学术研究工作的能力,专业学位申请人应当具有独立承受专业实践工作的能力;学术学位申请人应当在学术研究领域做出创新性成果,专业学位申请人应当在专业实践领域做出创新性成果。

    硕士学位论文表明作者在本学科或者专业领域掌握坚实的基础理论和系统的专门知识,学术学位申请人应当具有从事学术研究工作的能力,专业学位申请人应当具有承担专业实践工作的能力。

    为了提高研究生学位论文质量,进一步促进我校研究生学位论文的规范化,我们在原《南开大学研究生学位论文写作规范(试行)》的基础上,参考GB/T 7713.1-2006《学位论文编写规则》~\cite{SCSF00000378}等相关文件,对《写作规范》做了进一步修订,供申请学位的研究生参考,以利于学位论文的撰写、收藏、存储、加工、检索和利用。

    本规范适用于印刷型、缩微型、电子版、网络版等形式的学位论文。
    同一论文的不同载体形式,其内容和格式应完全一致。

    本规范是一个指导性规范,各一级学科学位评定分委员会可在参考本规范基础上,针对不同学科的学位类型的培养要求,分别制订相应的学位论文写作的具体规范(含外语类学位论文)。

    本规范共分四章,分别为1 内容要求,2 格式要求,3 书写要求,4 排版及印刷要求,并有附录A-G。
\end{preface}

\tableofcontents
\listoffigures
\listoftables
\begin{symabbr}
    符号、标志、缩略语、首字母缩写、计量单位、自定义名词和术语等的注释说明,如需汇集,应编写成注释说明汇集表,可集中置于图表清单之后。
    若上述符号使用数量不多,可以不设此部分,但必须在论文中出现时加以说明。
\end{symabbr}


\mainmatter

\chapter{内容要求}

研究生学位论文(thesis,dissertation)一般应以中文(简体汉字)撰写,外国语言文学学科学位论文可以用相应语种撰写。
其它学科专业如以英文撰写学位论文,需经学位评定分委员会批准后报研究生院备案。

鼓励使用外国语言接受研究生教学的学生(含国际学生)用中文撰写学位论文,如用教学过程中使用的外国语言进行撰写,需经学位评定分委员会批准后报研究生院备案。

用外文撰写的学位论文,须有详细中文摘要,英文摘要300-800英文实词。
目录之前的部分(包括封面、题名页、原创性声明、授权书、中英文摘要)仍需使用本规范规定的(中文)版本,目录及之后的内容可以使用与正文相同的语言。

学位论文一般由以下几部分组成,依次为:

\begin{enumerate}
    \item 中文封面;
    \item 题名页;
    \item 学位论文原创性声明和非公开学位论文标注说明;
    \item 学位论文使用授权书;
    \item 中文摘要;
    \item Abstract;
    \item 序言或前言(如有);
    \item 目录;
    \item 图和附表清单(如有);
    \item 符号、标志、缩略语等的注释表(如有);
    \item 正文;
    \item 附录;
    \item 参考文献;
    \item 分类索引、关键词索引(如有);
    \item 勘误页(如有);
    \item 致谢;
    \item 个人简历、在学期间发表的学术论文及研究成果。 
\end{enumerate}

\chapter{格式要求}
\label{chap:format-requirements}

\section{中文封面}
\label{sec:format-cover}

封面(cover)是学位论文的外表面,对论文起装潢和保护作用,并提供相关的信息。
我校申请博士、硕士和硕士专业学位的学位论文封面分别使用统一规定的不同封面。
封面中除已固定的内容外,其他需要填写的内容要求如下:

\textbf{分类号:}暂空

\textbf{密级:}按GB/T 7156-2003《文献保密等级代码与标识》标注。
公开论文可以标注“公开”,也可不标注;非公开论文标注“限制”、“秘密”或“机密”。
根据《南开大学关于研究生学位论文收藏和利用管理办法》(南发字〔2009〕23号文件)的规定,非公开论文须经申请、批准方能标注论文的密级(限制、秘密或机密),同时还应注明相应的保密年限(具体标注要求见\ref{sec:writing-classification})。

\textbf{学校代码:}10055

\textbf{学号:}填写本人学(申请)号。

\textbf{论文题目(title,又称题名):}应以简明词语恰当、准确地反映出论文最重要的特定内容,一般不宜超过25字,必要时可加论文副标题。

论文题目通常由名词性短语构成,应尽量避免使用不常用缩略词、首字母缩写字、字符、代号和公式等。

如论文题目内容层次很多,难以简化时,可采用论文题目和论文副标题相结合的方法,其中副标题起补充、阐明题目的作用。

示例1:斑马鱼和人的造血相关基因以及表观遗传学调控基因——进化、表达谱和功能研究

示例2:阿片镇痛的调控机制研究:Delta型阿片肽受体转运的调控机理及功能

\textbf{培养单位:}指学位申请人所在学院(所)名称,应采用规范全称。
如哲学院、数学科学学院等。

\textbf{一级学科:}学科名称以国务院学位委员会颁布的《研究生教育学科专业目录(2022年)》为准。

\textbf{二级学科:}学科名称参照国务院学位委员会颁布的《授予博士、硕士学位和培养研究生的学科、专业目录》或我校自主设置学科名称。

\textbf{专业学位名称:}填写硕士专业学位名称。
如工商管理硕士、法律硕士等等。

\textbf{论文作者:}填写作者姓名。

\textbf{指导教师:}填写指导教师的姓名、职称(教授、研究员等)。

\textbf{论文完成时间(年月):}填写论文提交评审的时间。

\section{题名页}

题名页(title page)包含论文全部书目信息,单独成页(示例见附录A)。
主要内容规定如下:

\textbf{中图分类号、UDC:}暂空

\textbf{学校代码:}按照教育部批准的学校代码标注,应为“10055”。

\textbf{密级:}标注同中文封面密级要求(具体标注见\ref{sec:writing-classification})。

\textbf{学位授予单位名称和学位论文类型:}保持“南开大学”题字不变。
根据学位论文类型填写“硕士学位论文”或“博士学位论文”或“硕士专业学位论文”或“博士专业学位论文”。

\textbf{题名(即论文题目)和副题名(即论文副标题):}题名要求同论文题目,应中英文对照。英文题名在中文题名下方。
题名和副题名在整篇学位论文中的不同地方出现时,应保持一致。

\textbf{责任者:}责任者包括论文作者姓名,指导教师姓名、职称等。
如责任者姓名有必要附注汉语拼音时,遵照GB/T 16159-2012《汉语拼音正词法基本规则》著录。

\textbf{申请学位}
包括申请的学位类别和级别,学位类别标注包括以下门类:哲学、经济学、法学、教育学、文学、历史学、理学、工学、农学、医学、军事学、管理学、艺术学。
学位级别标注包括硕士、博士。如哲学硕士、管理学博士。
硕士专业学位直接标注其名称,如工商管理硕士、工程硕士等。

\textbf{培养单位:}同\ref{sec:format-cover}中说明。

\textbf{学科专业:}参照国务院学位委员会颁布的《研究生教育学科专业目录(2022年)》、《授予博士、硕士学位和培养研究生的学科、专业目录》填写。
硕士专业学位填写领域名称,如无领域不必填此项。

\textbf{研究方向:}指本学科专业范畴下的研究方向。

\textbf{论文完成时间:}同\ref{sec:format-cover}中说明。

\section{学位论文原创性声明和非公开学位论文标注说明}

本部分放在题名页之后另起页,内容见附录B,可直接将附录B复制到论文中,但要删除附录题目。
提交时学位论文原创性声明须有作者亲笔签名。
非公开学位论文标注说明须有南开大学学位评定委员会办公室盖章方为有效。

\section{学位论文使用授权书}

本部分放在学位论文原创性声明之后另起页,内容见附录C,可直接将附录C复制到论文中,但要删除附录题目。
提交时须有作者亲笔签名。

\section{中文摘要}

中文摘要是论文内容的简要陈述,是一篇具有独立性和完整性的短文,一般以第三人称语气写成,不加评论和补充的解释。
摘要具有自含性,即不阅读论文的全文,就能获得必要的信息。
摘要的内容应包括与论文等同的主要信息,供读者确定有无必要阅读全文,也可供二次文献采用。
摘要一般应说明研究工作的目的、研究方法、研究成果和结论,要突出本论文的创造性成果。

中文摘要力求语言精炼准确,一般字数为500-800字,篇幅以一页为宜。如需要,字数可以略多。

用外文撰写学位论文时,须有详细中文摘要。

摘要中不可出现图、表、化学方程式、非公知公用的符号和术语。

关键词在摘要内容后另起一行标明,一般3-5个,之间用分号分开。
关键词是为了便于做文献索引和检索工作而从论文中选取出来用以表示全文主题内容信息的单词或术语,应体现论文特色,具有语义性,在论文中有明确出处。
应尽量采用《汉语主题词表》或各专业主题词表提供的规范词。

\section{Abstract}

Abstract内容与中文摘要相对应。
一般不少于300个英文实词,篇幅以一页为宜。
如需要,字数可以略多。

\section{序言或前言(如有)}

学位论文的序言或前言一般是作者对本篇论文基本特征的简介,如说明研究工作的缘起、背景、主旨、目的、意义、编写体例,以及资助、支持、协作经过等。
这些内容也可以在正文引言(绪论)中说明。

\section{目录(目次)}

学位论文应有目录(目次)(table of contents)页,排在序言(或前言)之后,另起页。
目录是论文各章节标题的顺序列表,附有相应的起始页码。

\section{图和附表清单(如有)}

论文中如图表较多,可以分别列出清单置于目录页之后另起页。
图的清单应有序号、图题和页码。表的清单应有序号、表题和页码。

\section{符号、标志、缩略语等的注释表(如有)}

符号、标志、缩略语、首字母缩写、计量单位、自定义名词和术语等的注释说明,如需汇集,应编写成注释说明汇集表,可集中置于图表清单之后。
若上述符号使用数量不多,可以不设此部分,但必须在论文中出现时加以说明。

\section{正文}

正文是学位论文的主体部分,应从另页右页开始,每一章应另起页。
学位论文字数:建议博士学位论文10万字左右,硕士学位论文3万字左右。
在保证学位论文质量的前提下,各一级学科根据学科特点可自行规定论文字数。

\subsection{引言或绪言}

引言或绪言(第一章):包括研究的目的和意义,问题的提出,选题的背景,文献综述,研究方法,论文结构安排等。

\subsection{具体章节}

本部分是论文作者的研究内容,是论文的核心。
各章之间互相关联,符合逻辑顺序。

\subsection{引文标注}

论文中引用的文献的标注方法遵照GB/T 7714-2015 《信息与文献参考文献著录规则》,可采用顺序编码制,也可采用著者-出版年制,但全文必须统一。

\subsection{注释}

当论文中的字、词或短语,需要进一步加以说明,而又没有具体文献来源时,用注释(notes)。
注释一般在社会科学中用的较多。
由于论文篇幅较长,建议采用文中编号加当“脚注”的方式。
最好不采用文中编号加“尾注”。
涉及参考文献的注释同正文加注。

\subsection{结论}

结论(最后一章):是学位论文最终和总体的结论,应明确、精练、完整、准确,不是正文中各段的小结的简单重复。
论文的结论应包括论文的核心观点,着重阐述作者的创造性工作及所取得的研究成果在本学术领域的地位、作用和意义,交代研究工作的局限,提出未来工作的意见或建议。

\section{附录}

有些材料编入文章主体会有损于编排的条理性和逻辑性,或有碍于文章结构的紧凑和突出主题思想等,可将这些材料作为附录编排于全文的末尾。

附录放在正文之后另起页。
附录的序号用A,B,C,…系列,如附录A,附录B,…。
附录中的公式、图和表的编号分别用A1,A2,…系列;图A1,图A2,…系列;表A1,表A2,…系列。每个附录应有标题。

\section{参考文献}

为了反映论文的科学依据和作者尊重他人研究成果的严肃态度以及向读者提供有关信息的出处,应列出参考文献表。
参考文献表是文中引用的有具体文字来源的文献集合,其著录项目和著录格式遵照GB/T 7714-2015《信息与文献参考文献著录规则》的规定执行。
参考文献表中列出的一般应限于作者直接阅读过被引用的、发表在正式出版物上的文献。
私人通信和未公开发表的资料,一般不宜列入参考文献,可紧跟在引用的内容之后注释或标注在当页的下方。

参考文献表应置于正文和附录后,并另起页。也可根据需要,在每页“脚注”中列出或在每章正文部分之后加入本章参考文献。

\section{分类索引、关键词索引(如有)}

如果需要,可以在参考文献后编排分类索引、关键词索引表。

\section{勘误页(如有)}

学位论文如有勘误页,应另起页,放在参考文献和分类索引、关键词索引后。
在勘误页顶部应放置下列信息。
题名、副题名(如有)、作者名。

\section{致谢}

致谢是作者对该文章的形成作过贡献的组织或个人予以感谢的文字记载,语言要诚恳、恰当、简短。
致谢应另起页,放置在参考文献、分类/关键词索引和勘误页后。
包括国家科学基金,资助研究工作的奖学金基金、合同单位、资助或支持的企业、组织或个人;协助完成研究工作和提供便利条件的组织或个人;在研究工作中提出建议和提供帮助的人;给予转载和引用权的资料、图片、文献、研究和调查的所有者;其他应感谢的组织和个人。

\section{个人简历、在学期间发表的学术论文与研究成果}

个人简历包括出生年月日、获得学士、硕士学位的学校、时间等;学术论文研究成果按发表的时间顺序列出(已发表的列在前面,已接收待发表的放在后面);研究成果可以是在学期间参加的研究项目、申请的专利或获奖等。

\chapter{书写要求}
\label{chap:writing-requirements}

\section{文字、标点符号和数字}

学位论文应用中文(简体汉字)汉字书写,外国语言文学学科学位论文可以用相应语种撰写。

汉字的使用应严格执行国家的有关规定,除特殊需要外,不得使用已废除的繁体字、异体字等不规范汉字。
标点符号的用法应该以GB/T 15834-2011《标点符号用法》~\cite{SCSF00038090}为准。
数字用法应该以GB/T 15835-2011《出版物上数字用法的规定》~\cite{SCSF00036561}为准。

\section{密级表注}
\label{sec:writing-classification}

根据《南开大学研究生学位论文收藏和利用管理办法》(南发字〔2009〕23号文件)的规定,非公开学位论文标注,须经本人申请,导师同意和相关部门批准方能在论文封面上标注密级,其他未经批准认定的论文一律视为公开级,不得涉及国家秘密。

根据GB/T 7156-2003《文献保密等级代码与标识》~\cite{SCSF00000734},对各密级定义如下:

公开级:文献可在国内外发行和交换。未标密的论文均视为公开级。

限制级:文献内容不涉及国家秘密,但在一定时间内限制其交流和使用范围。

秘密级:文献内容涉及一般国家秘密。

机密级:文献内容涉及重要的国家秘密。

密级标注标志的组成是:从左向右按密级、标识符、保密期限的顺序排列。标识符为“★”。

各密级的最长保密年限及书写格式规定如下:

限制★ 2 年(最长 2 年,可少于 2 年)

秘密★ 10年(最长 10 年,可少于 10 年)

机密★ 20年(最长 20 年,可少于 20 年)

示例1:限制★1年。表明此文献为限制级,自产生之日起满1年后解密。

示例2:秘密★5年。表明此文献为秘密级,自产生之日起满5年后解密。

示例3:机密★10年。表明此文献为机密级,自产生之日起满10年后解密。

文献保密期限的确定、变更或解密应按照有关的国家保密规定及南开大学的有关保密规定执行。

\section{章、节(层次标题)}

论文正文可以根据需要划分为不同数量的章、节,章、节的划分,建议参照CY/T 35-2001《科技文献的章节编号方法》~~\cite{REF00000004}。
章、节标题要简短、明确,同一层次的标题应尽可能“排比”,即词(或词组)类型相同(或相近),意义相关,语气一致。

章、节标题可有两种模式,模式一和模式二。
同一学位论文只能使用其中一种模式,不能交叉使用两种模式。
同一单位(同一学科)应该统一采用一种模式。

模式一中多层次标题用阿拉伯数字连续编号;不同层次的数字之间用小圆点“.”相隔,末位数字后面不加点号,如“3.1.2”;多层次标题的序号均左顶格起排,与标题间隔1个字距。

章、节标题的两种模式规定如下:

模式一示例:

1 ××××(章标题,左对齐)

1.1 ××××(一级标题,左对齐)

1.1.1 ××××(二级标题,左对齐)

1.1.1.1 ××××(根据需要,也可设三级标题,左对齐)

模式二示例:

第一章 ××××(章标题,居中)

第一节 ××××(一级标题,居中)

一、××××(二级标题,左缩进2个汉字)

(一)××××(根据需要,也可设三级标题,左缩进2个汉字)

1.××××(根据需要,也可设四级标题,左缩进2个汉字)

(1)××××(根据需要,也可设五级标题,左缩进2个汉字)

\section{篇眉和页码}

篇眉从中文摘要开始,内容与该部分的标题相同;论文页眉奇偶页相同,各部分首页也要有页眉。

页码从第一章(引言)开始按阿拉伯数字(1,2,3,……)连续编排,之前的部分(从中文摘要、Abstract、目录等至正文第一章前)用大写罗马数字(Ⅰ,Ⅱ,Ⅲ,……)单独编排。

\section{有关图、表、表达式}

论文中图、表、表达式应注明出处(如有),自制的图、表应说明资料、数据来源。

\subsection{图}

图包括曲线图、构造图、示意图、框图、流程图、记录图、地图、照片等。

图要精选,应具有自明性,切忌与表及文字表述重复。

图要清楚,但坐标比例不要过分放大,同一图上不同曲线的点要分别用不同形状的标识符标出。
图中的术语、符号、单位等应与正文表述中所用一致。
图在文中的布局要合理,一般随文编排,先见文字后见图。

图序与图题:图序即图的编号,由“图”和从“1”开始的阿拉伯数字组成,图较多时,可分章编号。
如第三章第1个图的图序为“\ref{fig:example}”;图题即图的名称,应简明,置于图序之后,图序和图题间空1个字距,居中置于图的下方。
例如:

\begin{figure}[H]
    \centering
    \begin{subfigure}[b]{0.4\linewidth}
        \centering
        \includegraphics[width=\linewidth]{example-image-a}
        \caption{子图A}
        \label{fig:example-a}
    \end{subfigure}
    \quad
    \begin{subfigure}[b]{0.4\linewidth}
        \centering
        \includegraphics[width=\linewidth]{example-image-b}
        \caption{子图B}
        \label{fig:example-b}
    \end{subfigure}
    \caption{非线性构形状态转移过程示意图}
    \label{fig:example}
\end{figure}

\begin{figure}[H]
    \centering
    \includegraphics[width=0.4\linewidth]{example-image-c}
    \caption{示例图C}
    \label{fig:example-c}
\end{figure}

\subsection{表}

表应有自明性。
表中参数应标明量和单位的符号。
表一般随文排,先见相应文字,后见表。

表序与表题:表序即表的编号,由“表”和从“1”开始的阿拉伯数字组成,表较多时,可分章编号,如第三章第1个表的表序为“表3.1”;表题即表的名称,应简明,置于表序之后,表序和表题间空1个字距,居中置于表的上方。例如:

\begin{table}[H]
    \centering
    \caption{线性五杆结构各自由度随机反应数值特征}
    \label{tab:example}
    \mslinespread{1.2}\zihao{5}
    \begin{tabular}{c|c|c|c|c|c|c}
        \hline
        \multirowcell{2}[-7pt][c]{$x_{1}$                                                \\(m)} & \multicolumn{3}{c|}{$F_{x}$} & \multicolumn{3}{c}{$x_{2}$}                                             \\
        \cline{2-7}
                 & \makecell{均值                                                          \\(N)}                        & \makecell{标准差\\(N)}                      & 变异系数     & \makecell{均值\\(m)}    & \makecell{标准差\\(m)}   & 变异系数     \\
        \hline
        0.000000 & 0.000000     & 0.000000   & 0.000000 & 0.000000 & 0.000000 & 0.000000 \\
        0.000100 & 206.006806   & 150.245905 & 0.729325 & 0.000024 & 0.000013 & 0.541667 \\
        0.000200 & 412.013613   & 215.100090 & 0.522070 & 0.000049 & 0.000018 & 0.367347 \\
        0.000300 & 618.020419   & 266.613296 & 0.431399 & 0.000073 & 0.000022 & 0.301370 \\
        \hline
    \end{tabular}
\end{table}

表的编排,一般是内容和测试项目由左至右横读,数据依序竖读。

表的编排建议采用国际通用的三线表。

如某表需要转页接排时,在随后的各页上应重复表序。
表序后跟表题(可省略)和“(续)”,居中置于表上方,续表均应重复表头。

\subsection{表达式}

表达式主要指数字表达式,也包括文字表达式。
表达式需另行起排,并缩格书写,与周围文字留足够的空间区分开。
如有两个以上的表达式,应用从“1”开始的阿拉伯数字进行编号,并将编号置于圆括号内。
表达式的编号右端对齐,表达式与编号之间可用“…”连接。
表达式较多时,可分章编号。
如第三章第1个表达式:

当广义控制截面$\Theta$具有\ref{equ:example}的广义本构关系时,可定义如下的截面示性数

\begin{equation}
    \Phi\left(\Theta\right) =
    \begin{cases}
        0, & \text{if } E = E_{0} \\
        1, & \text{if } E = E_{1}
    \end{cases}
    \label{equ:example}
\end{equation}

较长的表达式需要转行时,应尽可能在“$=$”处回行,或者在“$+$”、“$-$”、“$\times$”、“$/$”等符号处回行,公式中分数线的横线,其长度应等于或略大于分子和分母中较长的一方。
如正文中书写分数,应尽量将其高度降低为一行。
如将分数线书写为“$/$”,将根号改为负指数。

\section{附录}

附录作为主体部分的补充(不是必需的)。
下列内容可作为附录编于论文后。

——为了整篇论文材料的完整,但编于正文又有损于编排的条理性和逻辑性,这一材料包括比正文更为详尽的信息、研究方法和技术更深入的叙述,以及对了解正文内容有用的补充信息等。

——由于篇幅过大或取材于复制品而不便于编入正文的材料。

——不便于编入正文的罕见珍贵资料。

——对一般读者并非必要阅读,但对本专业同行有参考价值的资料。

——正文中未被引用但被阅读或具有补充信息的文献。

——某些重要的原始数据、数学推导、结构图、统计表、计算机打印输出件等。

\section{参考文献}

参考文献表的标注方法可采用顺序编码制,也可采用著者-出版年制。
建议参照《信息与文献 参考文献著录规则》(GB/T 7714-2015,见附录G)的要求书写参考文献。

顺序编码制(numeric references method):参考文献表可按正文中引用的文献出现的先后顺序连续编码,并将序号置于正文中引用参考文献的部位方括号中(上标)。

引用单篇、一处引用多篇、多次引用同一篇(方括号外标引文页码)文献示例:

……的控制\textsuperscript{[235]};……的思想\textsuperscript{[236]}。
裴伟\textsuperscript{[248,83]}……,…的研究\textsuperscript{[256-257]}。
……产生的结果”\textsuperscript{[320]198}。
……和目标\textsuperscript{[320]345}。

著者-出版年制(first element and date method):参考文献表引用的文献按文种集中,可分为中文、日文、西文、俄文、其他文种5部分。
中文参考文献在前,外文参考文献在后,按著者字顺和出版年排序。
中文参考文献(含中译文献)可以按著者汉语拼音字顺排列,也可按著者的笔画笔顺排列。
外文参考文献表可以按著(作)者姓氏字母顺序排序。

示例:

...in the scences (Crane, 1972), ...by Stieg (1981)

参考文献:

CRANE D,1972. Invisible college[M]. Chicago: Univ. of Chicago Press.

Stieg M F,1981. The ... histirians[J]. College and Research Libraries, 42(6):549-560.

引用单篇、多著者、同一著者同年不同文献和多次引用同一文献(括号外上标引文页码)示例:

……的共识(张忠智,1997),……(刘毅 等,1990),……的方针”(裴丽生,1981a)。……更好的问题(裴丽生 等,1981b)。……的结果”(李伟,1996)\textsuperscript{1194},……和目标”(李伟,1996)\textsuperscript{354}。

参考文献的作者不超过3位时,全部列出;超过3位时,只列前3位,后面加“,等”或相应的外文;作者姓名之间用“,”分开。文后参考文献按顺序编码制书写参考文献见附录G。

几种主要参考文献著录格式如下:

\begin{enumerate}
    \item 专著:[序号] 主要责任者.题名:其他题名信息[文献类型标识/文献载体标识].其他责任者.版本项.出版地:出版者,出版年:引文页码[引用日期].获取和访问路径(电子资源必备).数字对象唯一标识符(电子资源必备).
    \item 连续出版物:[序号] 主要责任者.题名:其他题名信息[文献类型标识/文献载体标识].年,卷(期)-年,卷(期).出版地:出版者,出版年[引用日期].获取和访问路径(电子资源必备).数字对象唯一标识符(电子资源必备).
    \item 学位论文:[序号] 主要责任者.题名[D].大学所在城市:大学名称,出版年[引用日期].获取和访问路径(电子资源必备).数字对象唯一标识符(电子资源必备).
    \item 专利文献:[序号] 专利申请者或所有者.专利题名:专利号[P].公告日期或公开日期[引用日期].获取和访问路径(电子资源必备).数字对象唯一标识符(电子资源必备).
    \item 标准文献:[序号] 主要责任者.标准名称:标准号[S].出版地:出版者,出版年: 引文页码[引用日期].获取和访问路径(电子资源必备).数字对象唯一标识符(电子资源必备).
    \item 电子资源(不包括电子专著、电子连续出版物、电子学位论文、电子专利):[序号] 主要责任者.题名:其他题名信息[EB/OL].出版地:出版者,出版年:引文页码[引用日期]. 获取和访问路径(电子资源必备).数字对象唯一标识符(电子资源必备).
\end{enumerate}

\section{量和单位}

论文中使用的有关量和单位要执行GB 3100~3102-1993(国家技术监督局1993-12-27发布,1994-07-01实施,eqv. ISO 1000:1992)有关量和单位的规定。
量的符号一般为单个拉丁字母或希腊字母,并一律采用斜体(pH例外)。

为区别不同情况,可在量符号上附加角标。

在表达量值时,在公式、图、表和文字叙述中,一律使用单位的国际符号,且用正体。
单位符号与数值间要留适当间隙。

\chapter{排版及印刷要求}

本章规定了学位论文排版要求,供撰写论文时参考。

\section{纸张规格和页面设置}

\begin{table}[H]
    \centering
    \caption{纸张规格和页面设置要求}
    \mslinespread{1.2}\zihao{5}
    \begin{tabular}{l|l}
        \hline
             & \makecell{\textbf{排版说明}}            \\
        \hline
        纸张   & A4(210×297,90g规格),幅面白色              \\
        \hline
        页面设置 & 上、下3.8cm,左、右3.2cm,页眉、页脚3.0cm,装订线0cm \\
        \hline
        篇眉   & \makecell[cl]{宋体10.5磅(或五号)居中        \\Abstract部分用Times New Roman字体10.5磅(或五号)}    \\
        \hline
        页码   & 宋体10.5磅(或五号)居中                      \\
        \hline
    \end{tabular}
\end{table}

\section{中文封面排版说明}

中文封面已有固定版式,只须提供有关内容。此排版要求供印刷部门参考。

\begin{table}[H]
    \centering
    \caption{中文封面排版要求}
    \mslinespread{1.2}\zihao{5}
    \begin{tabular}{l|l}
        \hline
                    & \makecell{\textbf{排版说明}}                 \\
        \hline
        密级          & 宋体10.5磅(或五号)加粗                           \\
        \hline
        学号          & Times New Roman字体10.5磅(或五号)加粗            \\
        \hline
        论文题目        & 宋体22磅(或二号)加粗                             \\
        \hline
        培养单位        & \multirowcell{5}[0pt][cl]{宋体16磅(或三号)加粗}  \\
        \cline{1-1}
        一级学科        &                                          \\
        \cline{1-1}
        二级学科或专业学位名称 &                                          \\
        \cline{1-1}
        论文作者        &                                          \\
        \cline{1-1}
        指导教师        &                                          \\
        \hline
        南开大学研究生院    & \multirowcell{2}[0pt][cl]{宋体16磅(或三号)加粗居中 \\年月信息可用汉字或阿拉伯数字} \\
        \cline{1-1}
        年月          &                                          \\
        \hline
    \end{tabular}
\end{table}

\section{书脊排版说明}

论文封面的书脊用仿宋14磅(或四号),固定值行距16磅,段前段后0磅。
上方写题目,中间写作者姓名,下方写“南开大学”,距上下边界均为5cm左右。

\section{题名页排版要求}

\begin{table}[H]
    \centering
    \caption{题名页排版要求}
    \mslinespread{1.2}\zihao{5}
    \begin{tabular}{l|l}
        \hline
                         & \makecell{\textbf{排版说明}}               \\
        \hline
        中图分类号            & \multirowcell{4}[0pt][cl]{宋体10.5磅(或五号) \\英文用Times New Roman字体10.5磅(或五号)
        }                                                         \\
        \cline{1-1}
        UDC              &                                        \\
        \cline{1-1}
        学校代码             &                                        \\
        \cline{1-1}
        密级               &                                        \\
        \hline
        南开大学博/硕士(专业)学位论文 & 宋体22磅(或二号)                             \\
        \hline
        题名和副题名           & \makecell[cl]{宋体22磅(或二号),如果字数过多可适当调整   \\英文用Times New Roman字体22磅(或二号)}         \\
        \hline
        论文作者             & \multirowcell{8}[0pt][cl]{宋体12磅(或小四)}  \\
        \cline{1-1}
        指导教师             &                                        \\
        \cline{1-1}
        申请学位             &                                        \\
        \cline{1-1}
        培养单位             &                                        \\
        \cline{1-1}
        学科专业             &                                        \\
        \cline{1-1}
        研究方向             &                                        \\
        \cline{1-1}
        答辩委员会主席          &                                        \\
        \cline{1-1}
        评阅人              &                                        \\
        \hline
        南开大学研究生院         & \multirowcell{2}[0pt][cl]{宋体14磅(或四号)居中 \\年月信息可用汉字或阿拉伯数字}                                   \\
        \cline{1-1}
        年月               &                                        \\
        \hline
    \end{tabular}
\end{table}


\section{中、英文摘要排版要求}

\begin{table}[H]
    \centering
    \caption{中、英文摘要排版要求}
    \mslinespread{1.2}\zihao{5}
    \renewcommand\arraystretch{1.5}
    \begin{tabular}{l|l|l}
        \hline
             & \makecell{\textbf{中文摘要排版要求}}                        & \makecell{\textbf{英文摘要排版要求}} \\
        \hline
        标题   & \makecell[cl]{摘要:黑体18磅(或小二)加粗                                                      \\居中,单                                                      倍行距,段前24磅,段\\后18磅} &  \makecell[cl]{Abstract: Arial字体18磅(或小二)\\加粗居中,单倍行距,段前24磅,\\段后18磅}           \\
        \hline
        段落文字 & \multirowcell{2}[0pt][cl]{宋体12磅(或小四)                                               \\固定值行距20磅,段前段后0磅\\“关键词”三字加粗
        }    & \multirowcell{2}[0pt][cl]{Times New Roman字体12磅(或小四)                                \\固定值行距20磅,段前段后0磅\\“Key Words”两词加粗
        }                                                                                         \\
        \cline{1-1}
        关键词  &                                                     &                              \\
        \hline
    \end{tabular}
\end{table}

\section{目录排版要求}

\begin{table}[H]
    \centering
    \caption{目录排版要求}
    \mslinespread{1.2}\zihao{5}
    \begin{tabular}{l|l|l}
        \hline
               & \makecell{\textbf{示例}}                       & \makecell{\textbf{排版说明}}    \\
        \hline
        标题     & 目录                                           & \makecell[cl]{黑体16磅(或三号)加粗居 \\中,单倍行距,段前24磅,\\段后18磅}             \\
        \hline
        各章目录   & \makecell[cl]{3\hspace{\ccwd}格式要求 …………………5                                 \\第三章\hspace{\ccwd}格式要求 ……………5}&\makecell[cl]{宋体14磅(或四号),单倍行\\距,段前6磅,段后0磅,两\\端对齐,页码右对齐} \\
        \hline
        一级标题目录 & \makecell[cl]{3.5\hspace{\ccwd}有关图、表、表达式…10                                \\第五节\hspace{\ccwd}有关图、表、表达式…10}& \makecell[cl]{宋体12磅(或小四),单倍行\\距,段前6磅,段后0磅,两\\端对齐,页码右对齐,左\\缩进1个汉字符}   \\
        \hline
        二级标题目录 & \makecell[cl]{3.5.1\hspace{\ccwd}图……………………10                               \\一、图………………………10}&                      \makecell[cl]{宋体10.5磅(或五号),单倍\\行距,段前6磅,段后0磅,\\两端对齐,页码右对齐,左缩\\进2个汉字符} \\
        \hline
    \end{tabular}
\end{table}

\section{正文排版要求}

\begin{table}[H]
    \centering
    \caption{正文排版要求}
    \mslinespread{1.2}\zihao{5}
    \renewcommand\arraystretch{1.5}
    \begin{tabular}{l|l|l}
        \hline
                                        & \makecell{\textbf{示例}}                 & \makecell{\textbf{排版说明}}                           \\
        \hline
        \multirowcell{2}[0pt][cl]{章标题}  & 1\hspace{\ccwd}×××                     & \multirowcell{2}[0pt][cl]{黑体16磅(或三号)加粗居左,单倍行距,段前24 \\磅,段后18磅,章序号与章题目间空一个汉字符。\\模式二加粗居中。}             \\
                                        & \makecell{第一章\hspace{\ccwd}×××}        &                                                    \\
        \hline
        \multirowcell{2}[0pt][cl]{一级标题} & 1.1\hspace{\ccwd}×××                   & \multirowcell{2}[0pt][cl]{黑体14磅(或四号)加粗居左,单倍行距,段前24 \\磅,段后6磅,序号与题名间空一个汉字符。模式二\\加粗居中。} \\
                                        & \makecell{第一节\hspace{\ccwd}×××}        &                                                    \\
        \hline
        二级标题                            & \makecell[cl]{1.1.1\hspace{\ccwd}×××                                                        \\\hspace{2\ccwd}一、×××}& \makecell[cl]{黑体13磅居左,单倍行距,段前12磅,段后6磅,\\序号与题名间空一个汉字符。模式二左缩进2个汉字。}   \\
        \hline
        三级标题                            & \makecell[cl]{1.1.1.1\hspace{\ccwd}×××                                                      \\\hspace{2\ccwd}(一)×××}&                      \makecell[cl]{黑体12磅(或小四)居左,单倍行距,段前12磅,\\段后6磅,序号与题名间空一个汉字符。模式二左缩\\进2个汉字。} \\
        \hline
        段落文字                            & \makecell[cl]{\hspace{2\ccwd}×××××××,                                                       \\×××××××。\\
        \hspace{2\ccwd}××××××,                                                                                                        \\×××。}& \makecell[cl]{宋体12磅(或小四),英文用Times New Roman字体\\12磅(或小四),两端对齐书写,段落首行左缩进2\\个汉字符。固定值行距20磅(段落中有数学表达式\\时,可根据表达需要设置该段的行距),段前0磅,段\\后0磅。} \\
        \hline
        图序、图题                           & \makecell{图2.1\hspace{\ccwd}×××}       & \makecell[cl]{置于图的下方,宋体10.5磅(或五号)居中,单倍行            \\距,段前6磅,段后12磅,图序与图题文字之间空一\\个汉字符宽度。} \\
        \hline
        表序、表题                           & \makecell{表3.1\hspace{\ccwd}×××}       & \makecell[cl]{置于表的上方,宋体10.5磅(或五号)居中,单倍行            \\距,段前6磅,段后6磅,表序与表题文字之间空一\\个汉字符宽度。} \\
        \hline
        表达式                             & \makecell[cr]{……(3.2)}                 & \makecell[cl]{序号加圆括号,Times New Roman 10.5磅(或五号),   \\右对齐。} \\
        \hline
    \end{tabular}
\end{table}


\section{其他部分排版要求}

\begin{table}[H]
    \centering
    \caption{其他排版要求}
    \mslinespread{1.2}\zihao{5}
    \begin{tabular}{l|l}
        \hline
             & \makecell{\textbf{排版说明}}                                \\
        \hline
        \makecell[cl]{符号、标志、缩略语                                        \\等的注释表}   & \makecell[cl]{标题要求同各章标题。文字部分:宋体10.5磅(或五号),英文\\用Times New Roman字体10.5磅(或五号),固定值行距16磅,\\段前段后0磅}              \\
        \hline
        附录   & \makecell[cl]{标题要求同各章标题。文字部分:宋体12磅(或小四),英文用             \\Times New Roman字体12磅(或小四),两端对齐书写,段落首\\行左缩进2个汉字符。固定值行距20磅(段落中有数学表达式\\时,可根据表达需要设置该段的行距),段前0磅,段后0磅} \\
        \hline
        参考文献 & \makecell[cl]{标题要求同各章标题。文字部分:宋体10.5磅(或五号),英文            \\用Times New Roman字体10.5磅(或五号),固定值行距16磅,\\段前段后0磅}    \\
        \hline
        勘误页  & \multirowcell{2}[0pt][cl]{标题要求同各章标题。文字部分仿宋12磅(或小四),固定值行 \\距16磅,段前段后0磅}                      \\
        \cline{1-1}
        致谢   &                                                         \\
        \hline
        \makecell[cl]{个人简历、在学期间                                        \\发表的学术论文与研\\究成果} & \makecell[cl]{标题要求同各章标题。文字部分:宋体10.5磅(或五号),英文\\用Times New Roman字体10.5磅(或五号),固定值行距16磅,\\段前段后0磅,发表学术论文书写格式同参考文献} \\
        \hline
    \end{tabular}
\end{table}

\section{论文印刷及装订要求}

论文自中文摘要起双面印刷,之前部分单面印刷。如果论文因页码过少而不能印刷书脊时,可以单面印刷。

根据论文存档要求,论文必须用线装或热胶装订,不能使用金属钉装订。


\backmatter

\appendix
\chapter{附录标题}

附录作为主体部分的补充(不是必需的)。
下列内容可作为附录编于论文后。

——为了整篇论文材料的完整,但编于正文又有损于编排的条理性和逻辑性,这一材料包括比正文更为详尽的信息、研究方法和技术更深入的叙述,以及对了解正文内容有用的补充信息等。

——由于篇幅过大或取材于复制品而不便于编入正文的材料。

——不便于编入正文的罕见珍贵资料。

——对一般读者并非必要阅读,但对本专业同行有参考价值的资料。

——正文中未被引用但被阅读或具有补充信息的文献。

——某些重要的原始数据、数学推导、结构图、统计表、计算机打印输出件等。

\bibliographypage
\begin{errata}
    学位论文如有勘误页,应另起页,放在参考文献和分类索引、关键词索引
    后。
    在勘误页顶部应放置下列信息。
    题名、副题名(如有)、作者名。
\end{errata}

\begin{acknowledgements}
    致谢是作者对该文章的形成作过贡献的组织或个人予以感谢的文字记载,语言要诚恳、恰当、简短。
    致谢应另起页,放置在参考文献、分类/关键词索引和勘误页后。
    包括国家科学基金,资助研究工作的奖学金基金、合同单位、资助或支持的企业、组织或个人;协助完成研究工作和提供便利条件的组织或个人;在研究工作中提出建议和提供帮助的人;给予转载和引用权的资料、图片、文献、研究和调查的所有者;其他应感谢的组织和个人。
\end{acknowledgements}

\begin{resume}
    个人简历包括出生年月日、获得学士、硕士学位的学校、时间等;学术论文研究成果按发表的时间顺序列出(已发表的列在前面,已接收待发表的放在后面);研究成果可以是在学期间参加的研究项目、申请的专利或获奖等。
\end{resume}


\end{document}
