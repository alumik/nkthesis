\chapter{格式要求}
\label{chap:format-requirements}

学位论文每部分从新的一页开始,各部分要求如下:

\section{中文封面}
\label{sec:format-cover}

封面(cover)是学位论文的外表面,对论文起装潢和保护作用,并提供相关的信息。
我校申请博士、硕士和硕士专业学位的学位论文封面分别使用统一规定的不同封面。
封面中除已固定的内容外,其他需要填写的内容要求如下:

\textbf{分类号:}暂空

\textbf{密级:}按GB/T 7156-2003《文献保密等级代码与标识》~\cite{SCSF00000734}标注。
公开论文可以标注“公开”,也可不标注;非公开论文标注“限制”、“秘密”或“机密”。
根据《南开大学关于研究生学位论文收藏和利用管理办法》(南发字〔2009〕23号文件)的规定,非公开论文须经申请、批准方能标注论文的密级(限制、秘密或机密),同时还应注明相应的保密年限(具体标注要求见\ref{sec:writing-classification})。

\textbf{学校代码:}10055

\textbf{学号:}填写本人学(申请)号。

\textbf{论文题目(title,又称题名):}应以简明词语恰当、准确地反映出论文最重要的特定内容,一般不宜超过25字,必要时可加论文副标题。

论文题目通常由名词性短语构成,应尽量避免使用不常用缩略词、首字母缩写字、字符、代号和公式等。

如论文题目内容层次很多,难以简化时,可采用论文题目和论文副标题相结合的方法,其中副标题起补充、阐明题目的作用。

示例1:斑马鱼和人的造血相关基因以及表观遗传学调控基因——进化、表达谱和功能研究

示例2:阿片镇痛的调控机制研究:Delta型阿片肽受体转运的调控机理及功能

\textbf{培养单位:}指学位申请人所在学院(所)名称,应采用规范全称。
如哲学院、数学科学学院等。

\textbf{一级学科:}学科名称以国务院学位委员会颁布的《研究生教育学科专业目录(2022年)》为准。

\textbf{二级学科:}学科名称参照国务院学位委员会颁布的《授予博士、硕士学位和培养研究生的学科、专业目录》~\cite{REF00000002}或我校自主设置学科名称。

\textbf{专业学位名称:}填写硕士专业学位名称。
如工商管理硕士、法律硕士等等。

\textbf{论文作者:}填写作者姓名。

\textbf{指导教师:}填写指导教师的姓名、职称(教授、研究员等)。

\textbf{论文完成时间(年月):}填写论文提交评审的时间。

\section{题名页}

题名页(title page)包含论文全部书目信息,单独成页(示例见附录A)。
主要内容规定如下:

\textbf{中图分类号、UDC:}暂空

\textbf{学校代码:}按照教育部批准的学校代码标注,应为“10055”。

\textbf{密级:}标注同中文封面密级要求(具体标注见\ref{sec:writing-classification})。

\textbf{学位授予单位名称和学位论文类型:}保持“南开大学”题字不变。
根据学位论文类型填写“硕士学位论文”或“博士学位论文”或“硕士专业学位论文”或“博士专业学位论文”。

\textbf{题名(即论文题目)和副题名(即论文副标题):}题名要求同论文题目,应中英文对照。英文题名在中文题名下方。
题名和副题名在整篇学位论文中的不同地方出现时,应保持一致。

\textbf{责任者:}责任者包括论文作者姓名,指导教师姓名、职称等。
如责任者姓名有必要附注汉语拼音时,遵照GB/T 16159-2012《汉语拼音正词法基本规则》~\cite{SCSF00038519}著录。

\textbf{申请学位}
包括申请的学位类别和级别,学位类别标注包括以下门类:哲学、经济学、法学、教育学、文学、历史学、理学、工学、农学、医学、军事学、管理学、艺术学。
学位级别标注包括硕士、博士。如哲学硕士、管理学博士。
硕士专业学位直接标注其名称,如工商管理硕士、工程硕士等。

\textbf{培养单位:}同\ref{sec:format-cover}中说明。

\textbf{学科专业:}参照国务院学位委员会颁布的《研究生教育学科专业目录(2022年)》、《授予博士、硕士学位和培养研究生的学科、专业目录》~\cite{REF00000002}填写。
硕士专业学位填写领域名称,如无领域不必填此项。

\textbf{研究方向:}指本学科专业范畴下的研究方向。

\textbf{论文完成时间:}同\ref{sec:format-cover}中说明。

\section{学位论文原创性声明和非公开学位论文标注说明}

本部分放在题名页之后另起页,内容见附录B,可直接将附录B复制到论文中,但要删除附录题目。
提交时学位论文原创性声明须有作者亲笔签名。
非公开学位论文标注说明须有南开大学学位评定委员会办公室盖章方为有效。

\section{学位论文使用授权书}

本部分放在学位论文原创性声明之后另起页,内容见附录C,可直接将附录C复制到论文中,但要删除附录题目。
提交时须有作者亲笔签名。

\section{中文摘要}

中文摘要是论文内容的简要陈述,是一篇具有独立性和完整性的短文,一般以第三人称语气写成,不加评论和补充的解释。
摘要具有自含性,即不阅读论文的全文,就能获得必要的信息。
摘要的内容应包括与论文等同的主要信息,供读者确定有无必要阅读全文,也可供二次文献采用。
摘要一般应说明研究工作的目的、研究方法、研究成果和结论,要突出本论文的创造性成果。

中文摘要力求语言精炼准确,一般字数为500-800字,篇幅以一页为宜。如需要,字数可以略多。

用外文撰写学位论文时,须有详细中文摘要。

摘要中不可出现图、表、化学方程式、非公知公用的符号和术语。

关键词在摘要内容后另起一行标明,一般3-5个,之间用分号分开。
关键词是为了便于做文献索引和检索工作而从论文中选取出来用以表示全文主题内容信息的单词或术语,应体现论文特色,具有语义性,在论文中有明确出处。
应尽量采用《汉语主题词表》~\cite{REF00000003}或各专业主题词表提供的规范词。

\section{Abstract}

Abstract内容与中文摘要相对应。
一般不少于300个英文实词,篇幅以一页为宜。
如需要,字数可以略多。

\section{序言或前言(如有)}

学位论文的序言或前言一般是作者对本篇论文基本特征的简介,如说明研究工作的缘起、背景、主旨、目的、意义、编写体例,以及资助、支持、协作经过等。
这些内容也可以在正文引言(绪论)中说明。

\section{目录(目次)}

学位论文应有目录(目次)(table of contents)页,排在序言(或前言)之后,另起页。
目录是论文各章节标题的顺序列表,附有相应的起始页码。

\section{图和附表清单(如有)}

论文中如图表较多,可以分别列出清单置于目录页之后另起页。
图的清单应有序号、图题和页码。表的清单应有序号、表题和页码。

\section{符号、标志、缩略语等的注释表(如有)}

符号、标志、缩略语、首字母缩写、计量单位、自定义名词和术语等的注释说明,如需汇集,应编写成注释说明汇集表,可集中置于图表清单之后。
若上述符号使用数量不多,可以不设此部分,但必须在论文中出现时加以说明。

\section{正文}

正文是学位论文的主体部分,应从另页右页开始,每一章应另起页。
学位论文字数:建议博士学位论文10万字左右,硕士学位论文3万字左右。
在保证学位论文质量的前提下,各一级学科根据学科特点可自行规定论文字数。

\subsection{引言或绪言}

引言或绪言(第一章):包括研究的目的和意义,问题的提出,选题的背景,文献综述,研究方法,论文结构安排等。

\subsection{具体章节}

本部分是论文作者的研究内容,是论文的核心。
各章之间互相关联,符合逻辑顺序。

\subsection{引文标注}

论文中引用的文献的标注方法遵照GB/T 7714-2015 《信息与文献参考文献著录规则》~\cite{SCSF00045714},可采用顺序编码制,也可采用著者-出版年制,但全文必须统一。

\subsection{注释}

当论文中的字、词或短语,需要进一步加以说明,而又没有具体文献来源时,用注释(notes)。
注释一般在社会科学中用的较多。
由于论文篇幅较长,建议采用文中编号加当“脚注”的方式。
最好不采用文中编号加“尾注”。
涉及参考文献的注释同正文加注。

\subsection{结论}

结论(最后一章):是学位论文最终和总体的结论,应明确、精练、完整、准确,不是正文中各段的小结的简单重复。
论文的结论应包括论文的核心观点,着重阐述作者的创造性工作及所取得的研究成果在本学术领域的地位、作用和意义,交代研究工作的局限,提出未来工作的意见或建议。

\section{附录}

有些材料编入文章主体会有损于编排的条理性和逻辑性,或有碍于文章结构的紧凑和突出主题思想等,可将这些材料作为附录编排于全文的末尾。

附录放在正文之后另起页。
附录的序号用A,B,C,…系列,如附录A,附录B,…。
附录中的公式、图和表的编号分别用A1,A2,…系列;图A1,图A2,…系列;表A1,表A2,…系列。每个附录应有标题。

\section{参考文献}

为了反映论文的科学依据和作者尊重他人研究成果的严肃态度以及向读者提供有关信息的出处,应列出参考文献表。
参考文献表是文中引用的有具体文字来源的文献集合,其著录项目和著录格式遵照GB/T 7714-2015《信息与文献参考文献著录规则》~\cite{SCSF00045714}的规定执行。
参考文献表中列出的一般应限于作者直接阅读过被引用的、发表在正式出版物上的文献。
私人通信和未公开发表的资料,一般不宜列入参考文献,可紧跟在引用的内容之后注释或标注在当页的下方。

参考文献表应置于正文和附录后,并另起页。也可根据需要,在每页“脚注”中列出或在每章正文部分之后加入本章参考文献。

\section{分类索引、关键词索引(如有)}

如果需要,可以在参考文献后编排分类索引、关键词索引表。

\section{勘误页(如有)}

学位论文如有勘误页,应另起页,放在参考文献和分类索引、关键词索引后。
在勘误页顶部应放置下列信息。
题名、副题名(如有)、作者名。

\section{致谢}

致谢是作者对该文章的形成作过贡献的组织或个人予以感谢的文字记载,语言要诚恳、恰当、简短。
致谢应另起页,放置在参考文献、分类/关键词索引和勘误页后。
包括国家科学基金,资助研究工作的奖学金基金、合同单位、资助或支持的企业、组织或个人;协助完成研究工作和提供便利条件的组织或个人;在研究工作中提出建议和提供帮助的人;给予转载和引用权的资料、图片、文献、研究和调查的所有者;其他应感谢的组织和个人。

\section{个人简历、在学期间发表的学术论文与研究成果}

个人简历包括出生年月日、获得学士、硕士学位的学校、时间等;学术论文研究成果按发表的时间顺序列出(已发表的列在前面,已接收待发表的放在后面);研究成果可以是在学期间参加的研究项目、申请的专利或获奖等。
