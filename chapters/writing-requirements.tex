\chapter{书写要求}
\label{chap:writing-requirements}

\section{文字、标点符号和数字}

学位论文应用中文(简体汉字)汉字书写,外国语言文学学科学位论文可以用相应语种撰写。

汉字的使用应严格执行国家的有关规定,除特殊需要外,不得使用已废除的繁体字、异体字等不规范汉字。
标点符号的用法应该以GB/T 15834-2011《标点符号用法》~\cite{SCSF00038090}为准。
数字用法应该以GB/T 15835-2011《出版物上数字用法的规定》~\cite{SCSF00036561}为准。

\section{密级表注}
\label{sec:writing-classification}

根据《南开大学研究生学位论文收藏和利用管理办法》(南发字〔2009〕23号文件)的规定,非公开学位论文标注,须经本人申请,导师同意和相关部门批准方能在论文封面上标注密级,其他未经批准认定的论文一律视为公开级,不得涉及国家秘密。

根据GB/T 7156-2003《文献保密等级代码与标识》~\cite{SCSF00000734},对各密级定义如下:

公开级:文献可在国内外发行和交换。未标密的论文均视为公开级。

限制级:文献内容不涉及国家秘密,但在一定时间内限制其交流和使用范围。

秘密级:文献内容涉及一般国家秘密。

机密级:文献内容涉及重要的国家秘密。

密级标注标志的组成是:从左向右按密级、标识符、保密期限的顺序排列。标识符为“{\segoeui{}★}”。

各密级的最长保密年限及书写格式规定如下:

限制{\segoeui{}★} 2 年(最长 2 年,可少于 2 年)

秘密{\segoeui{}★} 10年(最长 10 年,可少于 10 年)

机密{\segoeui{}★} 20年(最长 20 年,可少于 20 年)

示例1:限制{\segoeui{}★}1年。表明此文献为限制级,自产生之日起满1年后解密。

示例2:秘密{\segoeui{}★}5年。表明此文献为秘密级,自产生之日起满5年后解密。

示例3:机密{\segoeui{}★}10年。表明此文献为机密级,自产生之日起满10年后解密。

文献保密期限的确定、变更或解密应按照有关的国家保密规定及南开大学的有关保密规定执行。

\section{章、节(层次标题)}

论文正文可以根据需要划分为不同数量的章、节,章、节的划分,建议参照CY/T 35-2001《科技文献的章节编号方法》~~\cite{REF00000004}。
章、节标题要简短、明确,同一层次的标题应尽可能“排比”,即词(或词组)类型相同(或相近),意义相关,语气一致。

章、节标题可有两种模式,模式一和模式二。
同一学位论文只能使用其中一种模式,不能交叉使用两种模式。
同一单位(同一学科)应该统一采用一种模式。

模式一中多层次标题用阿拉伯数字连续编号;不同层次的数字之间用小圆点“.”相隔,末位数字后面不加点号,如“3.1.2”;多层次标题的序号均左顶格起排,与标题间隔1个字距。

章、节标题的两种模式规定如下:

模式一示例:

1 ××××(章标题,左对齐)

1.1 ××××(一级标题,左对齐)

1.1.1 ××××(二级标题,左对齐)

1.1.1.1 ××××(根据需要,也可设三级标题,左对齐)

模式二示例:

第一章 ××××(章标题,居中)

第一节 ××××(一级标题,居中)

一、××××(二级标题,左缩进2个汉字)

(一)××××(根据需要,也可设三级标题,左缩进2个汉字)

1.××××(根据需要,也可设四级标题,左缩进2个汉字)

(1)××××(根据需要,也可设五级标题,左缩进2个汉字)

\section{篇眉和页码}

篇眉从中文摘要开始,内容与该部分的标题相同;论文页眉奇偶页相同,各部分首页也要有页眉。

页码从第一章(引言)开始按阿拉伯数字(1,2,3,……)连续编排,之前的部分(从中文摘要、Abstract、目录等至正文第一章前)用大写罗马数字(Ⅰ,Ⅱ,Ⅲ,……)单独编排。

\section{有关图、表、表达式}

论文中图、表、表达式应注明出处(如有),自制的图、表应说明资料、数据来源。

\subsection{图}

图包括曲线图、构造图、示意图、框图、流程图、记录图、地图、照片等。

图要精选,应具有自明性,切忌与表及文字表述重复。

图要清楚,但坐标比例不要过分放大,同一图上不同曲线的点要分别用不同形状的标识符标出。
图中的术语、符号、单位等应与正文表述中所用一致。
图在文中的布局要合理,一般随文编排,先见文字后见图。

图序与图题:图序即图的编号,由“图”和从“1”开始的阿拉伯数字组成,图较多时,可分章编号。
如第三章第1个图的图序为“\ref{fig:example}”;图题即图的名称,应简明,置于图序之后,图序和图题间空1个字距,居中置于图的下方。
例如:

\begin{figure}[H]
    \centering
    \begin{subfigure}[b]{0.4\linewidth}
        \centering
        \includegraphics[width=\linewidth]{example-image-a}
        \caption{子图A}
        \label{fig:example-a}
    \end{subfigure}
    \quad
    \begin{subfigure}[b]{0.4\linewidth}
        \centering
        \includegraphics[width=\linewidth]{example-image-b}
        \caption{子图B}
        \label{fig:example-b}
    \end{subfigure}
    \caption{非线性构形状态转移过程示意图}
    \label{fig:example}
\end{figure}

\begin{figure}[H]
    \centering
    \includegraphics[width=0.4\linewidth]{example-image-c}
    \caption{示例图C}
    \label{fig:example-c}
\end{figure}

\subsection{表}

表应有自明性。
表中参数应标明量和单位的符号。
表一般随文排,先见相应文字,后见表。

表序与表题:表序即表的编号,由“表”和从“1”开始的阿拉伯数字组成,表较多时,可分章编号,如第三章第1个表的表序为“表3.1”;表题即表的名称,应简明,置于表序之后,表序和表题间空1个字距,居中置于表的上方。例如:

\begin{table}[H]
    \centering
    \caption{线性五杆结构各自由度随机反应数值特征}
    \label{tab:example}
    \mslinespread{1.2}\zihao{5}
    \begin{tabular}{c|c|c|c|c|c|c}
        \hline
        \multirowcell{2}[-7pt][c]{$x_{1}$                                                \\(m)} & \multicolumn{3}{c|}{$F_{x}$} & \multicolumn{3}{c}{$x_{2}$}                                             \\
        \cline{2-7}
                 & \makecell{均值                                                          \\(N)}                        & \makecell{标准差\\(N)}                      & 变异系数     & \makecell{均值\\(m)}    & \makecell{标准差\\(m)}   & 变异系数     \\
        \hline
        0.000000 & 0.000000     & 0.000000   & 0.000000 & 0.000000 & 0.000000 & 0.000000 \\
        0.000100 & 206.006806   & 150.245905 & 0.729325 & 0.000024 & 0.000013 & 0.541667 \\
        0.000200 & 412.013613   & 215.100090 & 0.522070 & 0.000049 & 0.000018 & 0.367347 \\
        0.000300 & 618.020419   & 266.613296 & 0.431399 & 0.000073 & 0.000022 & 0.301370 \\
        \hline
    \end{tabular}
\end{table}

表的编排,一般是内容和测试项目由左至右横读,数据依序竖读。

表的编排建议采用国际通用的三线表。

如某表需要转页接排时,在随后的各页上应重复表序。
表序后跟表题(可省略)和“(续)”,居中置于表上方,续表均应重复表头。

\subsection{表达式}

表达式主要指数字表达式,也包括文字表达式。
表达式需另行起排,并缩格书写,与周围文字留足够的空间区分开。
如有两个以上的表达式,应用从“1”开始的阿拉伯数字进行编号,并将编号置于圆括号内。
表达式的编号右端对齐,表达式与编号之间可用“…”连接。
表达式较多时,可分章编号。
如第三章第1个表达式:

当广义控制截面$\Theta$具有\ref{equ:example}的广义本构关系时,可定义如下的截面示性数

\begin{equation}
    \Phi\left(\Theta\right) =
    \begin{cases}
        0, & \text{if } E = E_{0} \\
        1, & \text{if } E = E_{1}
    \end{cases}
    \label{equ:example}
\end{equation}

较长的表达式需要转行时,应尽可能在“$=$”处回行,或者在“$+$”、“$-$”、“$\times$”、“$/$”等符号处回行,公式中分数线的横线,其长度应等于或略大于分子和分母中较长的一方。
如正文中书写分数,应尽量将其高度降低为一行。
如将分数线书写为“$/$”,将根号改为负指数。

\section{附录}

附录作为主体部分的补充(不是必需的)。
下列内容可作为附录编于论文后。

——为了整篇论文材料的完整,但编于正文又有损于编排的条理性和逻辑性,这一材料包括比正文更为详尽的信息、研究方法和技术更深入的叙述,以及对了解正文内容有用的补充信息等。

——由于篇幅过大或取材于复制品而不便于编入正文的材料。

——不便于编入正文的罕见珍贵资料。

——对一般读者并非必要阅读,但对本专业同行有参考价值的资料。

——正文中未被引用但被阅读或具有补充信息的文献。

——某些重要的原始数据、数学推导、结构图、统计表、计算机打印输出件等。

\section{参考文献}

参考文献表的标注方法可采用顺序编码制,也可采用著者-出版年制。
建议参照《信息与文献 参考文献著录规则》(GB/T 7714-2015,见附录G)的要求书写参考文献。

顺序编码制(numeric references method):参考文献表可按正文中引用的文献出现的先后顺序连续编码,并将序号置于正文中引用参考文献的部位方括号中(上标)。

引用单篇、一处引用多篇、多次引用同一篇(方括号外标引文页码)文献示例:

……的控制\textsuperscript{[235]};……的思想\textsuperscript{[236]}。
裴伟\textsuperscript{[248,83]}……,…的研究\textsuperscript{[256-257]}。
……产生的结果”\textsuperscript{[320]198}。
……和目标\textsuperscript{[320]345}。

著者-出版年制(first element and date method):参考文献表引用的文献按文种集中,可分为中文、日文、西文、俄文、其他文种5部分。
中文参考文献在前,外文参考文献在后,按著者字顺和出版年排序。
中文参考文献(含中译文献)可以按著者汉语拼音字顺排列,也可按著者的笔画笔顺排列。
外文参考文献表可以按著(作)者姓氏字母顺序排序。

示例:

...in the scences (Crane, 1972), ...by Stieg (1981)

参考文献:

CRANE D,1972. Invisible college[M]. Chicago: Univ. of Chicago Press.

Stieg M F,1981. The ... histirians[J]. College and Research Libraries, 42(6):549-560.

引用单篇、多著者、同一著者同年不同文献和多次引用同一文献(括号外上标引文页码)示例:

……的共识(张忠智,1997),……(刘毅 等,1990),……的方针”(裴丽生,1981a)。……更好的问题(裴丽生 等,1981b)。……的结果”(李伟,1996)\textsuperscript{1194},……和目标”(李伟,1996)\textsuperscript{354}。

参考文献的作者不超过3位时,全部列出;超过3位时,只列前3位,后面加“,等”或相应的外文;作者姓名之间用“,”分开。文后参考文献按顺序编码制书写参考文献见附录G。

几种主要参考文献著录格式如下:

\begin{enumerate}
    \item 专著:[序号] 主要责任者.题名:其他题名信息[文献类型标识/文献载体标识].其他责任者.版本项.出版地:出版者,出版年:引文页码[引用日期].获取和访问路径(电子资源必备).数字对象唯一标识符(电子资源必备).
    \item 连续出版物:[序号] 主要责任者.题名:其他题名信息[文献类型标识/文献载体标识].年,卷(期)-年,卷(期).出版地:出版者,出版年[引用日期].获取和访问路径(电子资源必备).数字对象唯一标识符(电子资源必备).
    \item 学位论文:[序号] 主要责任者.题名[D].大学所在城市:大学名称,出版年[引用日期].获取和访问路径(电子资源必备).数字对象唯一标识符(电子资源必备).
    \item 专利文献:[序号] 专利申请者或所有者.专利题名:专利号[P].公告日期或公开日期[引用日期].获取和访问路径(电子资源必备).数字对象唯一标识符(电子资源必备).
    \item 标准文献:[序号] 主要责任者.标准名称:标准号[S].出版地:出版者,出版年: 引文页码[引用日期].获取和访问路径(电子资源必备).数字对象唯一标识符(电子资源必备).
    \item 电子资源(不包括电子专著、电子连续出版物、电子学位论文、电子专利):[序号] 主要责任者.题名:其他题名信息[EB/OL].出版地:出版者,出版年:引文页码[引用日期]. 获取和访问路径(电子资源必备).数字对象唯一标识符(电子资源必备).
\end{enumerate}

\section{量和单位}

论文中使用的有关量和单位要执行GB 3100~3102-1993(国家技术监督局1993-12-27发布,1994-07-01实施,eqv. ISO 1000:1992)有关量和单位的规定。
量的符号一般为单个拉丁字母或希腊字母,并一律采用斜体(pH例外)。

为区别不同情况,可在量符号上附加角标。

在表达量值时,在公式、图、表和文字叙述中,一律使用单位的国际符号,且用正体。
单位符号与数值间要留适当间隙。
