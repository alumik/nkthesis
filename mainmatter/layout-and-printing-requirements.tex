\chapter{排版及印刷要求}

本章规定了学位论文排版要求,供撰写论文时参考。

\section{纸张规格和页面设置}

\begin{table}[H]
    \centering
    \caption{纸张规格和页面设置要求}
    \mslinespread{1.2}\zihao{5}
    \begin{tabular}{l|l}
        \hline
             & \makecell{\heiti 排版说明}            \\
        \hline
        纸张   & A4(210×297,90g规格),幅面白色              \\
        \hline
        页面设置 & 上、下3.8cm,左、右3.2cm,页眉、页脚3.0cm,装订线0cm \\
        \hline
        篇眉   & \makecell[cl]{宋体10.5磅(或五号)居中        \\Abstract部分用Times New Roman字体10.5磅(或五号)}    \\
        \hline
        页码   & 宋体10.5磅(或五号)居中                      \\
        \hline
    \end{tabular}
\end{table}

\section{中文封面排版说明}

中文封面已有固定版式,只须提供有关内容。此排版要求供印刷部门参考。

\begin{table}[H]
    \centering
    \caption{中文封面排版要求}
    \mslinespread{1.2}\zihao{5}
    \begin{tabular}{l|l}
        \hline
                    & \makecell{\heiti 排版说明}                 \\
        \hline
        密级          & 宋体10.5磅(或五号)加粗                           \\
        \hline
        学号          & Times New Roman字体10.5磅(或五号)加粗            \\
        \hline
        论文题目        & 宋体22磅(或二号)加粗                             \\
        \hline
        培养单位        & \multirowcell{5}[0pt][cl]{宋体16磅(或三号)加粗}  \\
        \cline{1-1}
        一级学科        &                                          \\
        \cline{1-1}
        二级学科或专业学位名称 &                                          \\
        \cline{1-1}
        论文作者        &                                          \\
        \cline{1-1}
        指导教师        &                                          \\
        \hline
        南开大学研究生院    & \multirowcell{2}[0pt][cl]{宋体16磅(或三号)加粗居中 \\年月信息可用汉字或阿拉伯数字} \\
        \cline{1-1}
        年月          &                                          \\
        \hline
    \end{tabular}
\end{table}

\section{书脊排版说明}

论文封面的书脊用仿宋14磅(或四号),固定值行距16磅,段前段后0磅。
上方写题目,中间写作者姓名,下方写“南开大学”,距上下边界均为5cm左右。

\section{题名页排版要求}

\begin{table}[H]
    \centering
    \caption{题名页排版要求}
    \mslinespread{1.2}\zihao{5}
    \begin{tabular}{l|l}
        \hline
                         & \makecell{\heiti 排版说明}               \\
        \hline
        中图分类号            & \multirowcell{4}[0pt][cl]{宋体10.5磅(或五号) \\英文用Times New Roman字体10.5磅(或五号)
        }                                                         \\
        \cline{1-1}
        UDC              &                                        \\
        \cline{1-1}
        学校代码             &                                        \\
        \cline{1-1}
        密级               &                                        \\
        \hline
        南开大学博/硕士(专业)学位论文 & 宋体22磅(或二号)                             \\
        \hline
        题名和副题名           & \makecell[cl]{宋体22磅(或二号),如果字数过多可适当调整   \\英文用Times New Roman字体22磅(或二号)}         \\
        \hline
        论文作者             & \multirowcell{8}[0pt][cl]{宋体12磅(或小四)}  \\
        \cline{1-1}
        指导教师             &                                        \\
        \cline{1-1}
        申请学位             &                                        \\
        \cline{1-1}
        培养单位             &                                        \\
        \cline{1-1}
        学科专业             &                                        \\
        \cline{1-1}
        研究方向             &                                        \\
        \cline{1-1}
        答辩委员会主席          &                                        \\
        \cline{1-1}
        评阅人              &                                        \\
        \hline
        南开大学研究生院         & \multirowcell{2}[0pt][cl]{宋体14磅(或四号)居中 \\年月信息可用汉字或阿拉伯数字}                                   \\
        \cline{1-1}
        年月               &                                        \\
        \hline
    \end{tabular}
\end{table}


\section{中、英文摘要排版要求}

\begin{table}[H]
    \centering
    \caption{中、英文摘要排版要求}
    \mslinespread{1.2}\zihao{5}
    \renewcommand\arraystretch{1.5}
    \begin{tabular}{l|l|l}
        \hline
             & \makecell{\heiti 中文摘要排版要求}                        & \makecell{\heiti 英文摘要排版要求} \\
        \hline
        标题   & \makecell[cl]{摘要:黑体18磅(或小二)加粗                                                      \\居中,单                                                      倍行距,段前24磅,段\\后18磅} &  \makecell[cl]{Abstract: Arial字体18磅(或小二)\\加粗居中,单倍行距,段前24磅,\\段后18磅}           \\
        \hline
        段落文字 & \multirowcell{2}[0pt][cl]{宋体12磅(或小四)                                               \\固定值行距20磅,段前段后0磅\\“\textbf{关键词}”三字加粗
        }    & \multirowcell{2}[0pt][cl]{Times New Roman字体12磅(或小四)                                \\固定值行距20磅,段前段后0磅\\“\textbf{Key Words}”两词加粗
        }                                                                                         \\
        \cline{1-1}
        关键词  &                                                     &                              \\
        \hline
    \end{tabular}
\end{table}

\section{目录排版要求}

\begin{table}[H]
    \centering
    \caption{目录排版要求}
    \mslinespread{1.2}\zihao{5}
    \begin{tabular}{l|l|l}
        \hline
               & \makecell{\heiti 示例}                       & \makecell{\heiti 排版说明}    \\
        \hline
        标题     & 目录                                           & \makecell[cl]{黑体16磅(或三号)加粗居 \\中,单倍行距,段前24磅,\\段后18磅}             \\
        \hline
        各章目录   & \makecell[cl]{3\hspace{\ccwd}格式要求 …………………5                                 \\第三章\hspace{\ccwd}格式要求 ……………5}&\makecell[cl]{宋体14磅(或四号),单倍行\\距,段前6磅,段后0磅,两\\端对齐,页码右对齐} \\
        \hline
        一级标题目录 & \makecell[cl]{3.5\hspace{\ccwd}有关图、表、表达式…10                                \\第五节\hspace{\ccwd}有关图、表、表达式…10}& \makecell[cl]{宋体12磅(或小四),单倍行\\距,段前6磅,段后0磅,两\\端对齐,页码右对齐,左\\缩进1个汉字符}   \\
        \hline
        二级标题目录 & \makecell[cl]{3.5.1\hspace{\ccwd}图……………………10                               \\一、图………………………10}&                      \makecell[cl]{宋体10.5磅(或五号),单倍\\行距,段前6磅,段后0磅,\\两端对齐,页码右对齐,左缩\\进2个汉字符} \\
        \hline
    \end{tabular}
\end{table}

\section{正文排版要求}

\begin{table}[H]
    \centering
    \caption{正文排版要求}
    \mslinespread{1.2}\zihao{5}
    \renewcommand\arraystretch{1.5}
    \begin{tabular}{l|l|l}
        \hline
                                        & \makecell{\heiti 示例}                 & \makecell{\heiti 排版说明}                           \\
        \hline
        \multirowcell{2}[0pt][cl]{章标题}  & 1\hspace{\ccwd}×××                     & \multirowcell{2}[0pt][cl]{黑体16磅(或三号)加粗居左,单倍行距,段前24 \\磅,段后18磅,章序号与章题目间空一个汉字符。\\模式二加粗居中。}             \\
                                        & \makecell{第一章\hspace{\ccwd}×××}        &                                                    \\
        \hline
        \multirowcell{2}[0pt][cl]{一级标题} & 1.1\hspace{\ccwd}×××                   & \multirowcell{2}[0pt][cl]{黑体14磅(或四号)加粗居左,单倍行距,段前24 \\磅,段后6磅,序号与题名间空一个汉字符。模式二\\加粗居中。} \\
                                        & \makecell{第一节\hspace{\ccwd}×××}        &                                                    \\
        \hline
        二级标题                            & \makecell[cl]{1.1.1\hspace{\ccwd}×××                                                        \\\hspace{2\ccwd}一、×××}& \makecell[cl]{黑体13磅居左,单倍行距,段前12磅,段后6磅,\\序号与题名间空一个汉字符。模式二左缩进2个汉字。}   \\
        \hline
        三级标题                            & \makecell[cl]{1.1.1.1\hspace{\ccwd}×××                                                      \\\hspace{2\ccwd}(一)×××}&                      \makecell[cl]{黑体12磅(或小四)居左,单倍行距,段前12磅,\\段后6磅,序号与题名间空一个汉字符。模式二左缩\\进2个汉字。} \\
        \hline
        段落文字                            & \makecell[cl]{\hspace{2\ccwd}×××××××,                                                       \\×××××××。\\
        \hspace{2\ccwd}××××××,                                                                                                        \\×××。}& \makecell[cl]{宋体12磅(或小四),英文用Times New Roman字体\\12磅(或小四),两端对齐书写,段落首行左缩进2\\个汉字符。固定值行距20磅(段落中有数学表达式\\时,可根据表达需要设置该段的行距),段前0磅,段\\后0磅。} \\
        \hline
        图序、图题                           & \makecell{图2.1\hspace{\ccwd}×××}       & \makecell[cl]{置于图的下方,宋体10.5磅(或五号)居中,单倍行            \\距,段前6磅,段后12磅,图序与图题文字之间空一\\个汉字符宽度。} \\
        \hline
        表序、表题                           & \makecell{表3.1\hspace{\ccwd}×××}       & \makecell[cl]{置于表的上方,宋体10.5磅(或五号)居中,单倍行            \\距,段前6磅,段后6磅,表序与表题文字之间空一\\个汉字符宽度。} \\
        \hline
        表达式                             & \makecell[cr]{……(3.2)}                 & \makecell[cl]{序号加圆括号,Times New Roman 10.5磅(或五号),   \\右对齐。} \\
        \hline
    \end{tabular}
\end{table}


\section{其他部分排版要求}

\begin{table}[H]
    \centering
    \caption{其他排版要求}
    \mslinespread{1.2}\zihao{5}
    \begin{tabular}{l|l}
        \hline
             & \makecell{\heiti 排版说明}                                \\
        \hline
        \makecell[cl]{符号、标志、缩略语                                        \\等的注释表}   & \makecell[cl]{标题要求同各章标题。文字部分:宋体10.5磅(或五号),英文\\用Times New Roman字体10.5磅(或五号),固定值行距16磅,\\段前段后0磅}              \\
        \hline
        附录   & \makecell[cl]{标题要求同各章标题。文字部分:宋体12磅(或小四),英文用             \\Times New Roman字体12磅(或小四),两端对齐书写,段落首\\行左缩进2个汉字符。固定值行距20磅(段落中有数学表达式\\时,可根据表达需要设置该段的行距),段前0磅,段后0磅} \\
        \hline
        参考文献 & \makecell[cl]{标题要求同各章标题。文字部分:宋体10.5磅(或五号),英文            \\用Times New Roman字体10.5磅(或五号),固定值行距16磅,\\段前段后0磅}    \\
        \hline
        勘误页  & \multirowcell{2}[0pt][cl]{标题要求同各章标题。文字部分仿宋12磅(或小四),固定值行 \\距16磅,段前段后0磅}                      \\
        \cline{1-1}
        致谢   &                                                         \\
        \hline
        \makecell[cl]{个人简历、在学期间                                        \\发表的学术论文与研\\究成果} & \makecell[cl]{标题要求同各章标题。文字部分:宋体10.5磅(或五号),英文\\用Times New Roman字体10.5磅(或五号),固定值行距16磅,\\段前段后0磅,发表学术论文书写格式同参考文献} \\
        \hline
    \end{tabular}
\end{table}

\section{论文印刷及装订要求}

论文自中文摘要起双面印刷,之前部分单面印刷。如果论文因页码过少而不能印刷书脊时,可以单面印刷。

根据论文存档要求,论文必须用线装或热胶装订,不能使用金属钉装订。
